\documentclass{article}

\title{Chapter 10: Rotation}
\author{Rylan Polster}

\usepackage[parfill]{parskip}
\usepackage{multirow}
\usepackage{amsmath,amssymb,amsthm}
\usepackage{bm}
\usepackage{textcomp,gensymb}
\usepackage{siunitx}
\usepackage[margin=1.0in]{geometry}

\begin{document}
    \maketitle

    \section*{General}

        \paragraph{Quantities}
        \begin{align}
            s &= \text{arc length} \nonumber\\
            r &= \text{radius} \nonumber\\
            \theta &= \text{angular position} \nonumber\\
            \omega &= \text{angular velocity} \nonumber\\
            \alpha &= \text{angular acceleration} \nonumber\\
            v &= \text{linear velocity} \nonumber\\
            a_t &= \text{tangential acceleration} \nonumber\\
            a_r &= \text{radial acceleration} \nonumber\\
            T &= \text{period of motion} \nonumber\\
            K &= \text{kinetic energy} \nonumber\\
            I &= \text{rotational inertia} \nonumber\\
            I_\text{com} &= \text{rotational inertia about an axis at the center of mass} \nonumber\\
            h &= \text{perpendicular distance between two axes} \nonumber\\
            \tau &= \text{torque} \nonumber\\
            \phi &= \text{angle between the force and the lever arm} \nonumber
        \end{align}

    \section{Rotational Variables}

        \paragraph{Angular Position}
        \begin{equation}
            \theta = \frac{s}{r}
        \end{equation}

        \paragraph{Angular Velocity}
        \begin{equation}
            \omega = \lim_{\Delta t \to 0} \frac{\Delta \theta}{\Delta t} = \frac{d\theta}{dt}
        \end{equation}

        \paragraph{Angular Acceleration}
        \begin{equation}
            \alpha = \lim_{\Delta t \to 0} \frac{\Delta \omega}{\Delta t} = \frac{d\omega}{dt}
        \end{equation}

    \section{Rotation With Constant Angular Acceleration}

        \begin{equation}
            \omega = \omega_0 + \alpha t
        \end{equation}
        \begin{equation}
            \theta - \theta_0 = \omega_0 t + \frac{1}{2} \alpha t^2
        \end{equation}
        \begin{equation}
            \omega^2 = \omega_0^2 + 2 \alpha \left( \theta - \theta_0 \right)
        \end{equation}
        \begin{equation}
            \theta - \theta_0 = \frac{1}{2} \left( \omega_0 + \omega \right) t
        \end{equation}
        \begin{equation}
            \theta - \theta_0 = \omega t - \frac{1}{2} \alpha t^2
        \end{equation}

    \section{Relating the Linear and Angular Variables}

        \paragraph{Position}
        \begin{equation}
            s = \theta r
        \end{equation}

        \paragraph{Speed}
        \begin{equation}
            v = \omega r
        \end{equation}

        \paragraph{Tangential Acceleration}
        \begin{equation}
            a_t = \alpha r
        \end{equation}

        \paragraph{Radial Acceleration}
        \begin{equation}
            a_r = \frac{v^2}{r} = \omega^2 r
        \end{equation}
        Radial acceleration is directed toward the center

        \paragraph{Period of Motion}
        \begin{eqnarray}
            T = \frac{2 \pi r}{v} = \frac{2 \pi}{\omega}
        \end{eqnarray}

    \section{Kinetic Energy of Rotation}

        \paragraph{Rotation Inertia}
        \begin{equation}
            I = \sum m_i r_i^2
        \end{equation}

        \paragraph{Rotational Kinetic Energy}
        \begin{equation}
            K = \frac{1}{2} I \omega^2
        \end{equation}

    \section{Calculating the Rotational Inertia}

        \begin{eqnarray}
            I = \int r^2 \, dm
        \end{eqnarray}

        \paragraph{Parallel-Axis Theorem}
        \begin{equation}
            I = I_\text{com} + M h^2
        \end{equation}

    \section{Torque}

        \begin{equation}
            \tau = \left( r \right) \left( F \sin{\phi} \right)
        \end{equation}

        There are two (equivalent) ways to calculate torque:

        \begin{equation}
            \tau = \left( r \right) \left( F \sin{\phi} \right) = r F_t
        \end{equation}
        \begin{equation}
            \tau = \left( r \sin{\phi} \right) \left( F \right) = r_\perp F
        \end{equation}


\end{document}
