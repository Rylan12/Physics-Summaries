\documentclass{article}

\title{Chapter 11: Rolling, Torque, and Angular Momentum}
\author{Rylan Polster}

\usepackage[parfill]{parskip}
\usepackage{multirow}
\usepackage{amsmath,amssymb,amsthm}
\usepackage{bm}
\usepackage{textcomp,gensymb}
\usepackage{siunitx}
\usepackage[margin=1.0in]{geometry}

\begin{document}
    \maketitle

    \section*{General}

        \paragraph{Quantities}
        \begin{align}
            v &= \text{linear velocity} \nonumber\\
            \omega &= \text{angular velocity} \nonumber\\
            R &= \text{radius} \nonumber\\
            K &= \text{kinetic energy} \nonumber\\
            I &= \text{rotational inertia} \nonumber\\
            M &= \text{mass} \nonumber\\
            \tau &= \text{torque} \nonumber\\
            \vec{r} &= \text{lever arm} \nonumber\\
            \vec{F} &= \text{force} \nonumber\\
            \vec{\ell} &= \text{angular momentum} \nonumber\\
            \vec{L} &= \text{angular momentum of a system} \nonumber\\
            \vec{v} &= \text{linear momentum} \nonumber
        \end{align}

    \section{Rolling As Translation and Rotation Combined}

        \paragraph{Center of Wheel}
        \begin{equation}
            v_\text{com} = \omega R
        \end{equation}

        \paragraph{Top of Wheel}
        \begin{equation}
            v_\text{top} = 2 v_\text{com}
        \end{equation}

        \paragraph{Bottom of Wheel}
        \begin{equation}
            v_\text{bottom} = 0
        \end{equation}

    \section{Kinetic Energy of A Rolling Object}

        \begin{equation}
            K = \frac{1}{2} I_\text{com} \omega^2 + \frac{1}{2} M v_\text{com}^2
        \end{equation}

    \section{Torque}

        \begin{equation}
            \vec{\tau} = \vec{r} \times \vec{F}
        \end{equation}

    \section{Angular Momentum}

        \begin{equation}
            \vec{\ell} = \vec{r} \times \vec{p} = m \left( \vec{r} \times \vec{v} \right)
        \end{equation}

        \paragraph{Newton's Second Law in Angular Momentum}
        The (vector) sum of all the torques acting on a particle is equal to the time rate of change of the angular momentum of that particle.
        \begin{equation}
            \vec{\tau}_\text{net} = \frac{d\vec{\ell}}{dt}
        \end{equation}

        \paragraph{Angular Momentum of A Rigid Body}
        \begin{equation}
            \vec{\tau} = \frac{d\vec{L}}{dt}
        \end{equation}
        \begin{equation}
            L = I \omega
        \end{equation}

    \section{Conservation of Angular Momentum}

        If the net external torque acting on a system is zero, the angular momentum $\vec{L}$ of the system remains constant, no matter what changes take place within the system.

        \begin{equation}
            \vec{L}_i = \vec{L}_f
        \end{equation}

\end{document}
