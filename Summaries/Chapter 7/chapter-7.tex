\documentclass{article}

\title{Chapter 7: Kinetic Energy and Work}
\author{Rylan Polster}

\usepackage[parfill]{parskip}
\usepackage{multirow}
\usepackage{amsmath,amssymb,amsthm}
\usepackage{bm}
\usepackage{textcomp,gensymb}
\usepackage{siunitx}
\usepackage[margin=1.0in]{geometry}

\begin{document}
    \maketitle
    
    \section*{General}

        \paragraph{Quantities}
        \begin{align}
            K &= \text{kinetic energy} \nonumber\\
            W &= \text{work} \nonumber\\
            \vec{F} &= \text{force} \nonumber\\
            m &= \text{mass} \nonumber\\
            g &= \text{magnitude of free-fall acceleration} \nonumber\\
            v &= \text{velocity} \nonumber\\
            k &= \text{force constant/spring constant} \nonumber\\
            \vec{d} &= \text{displacement} \nonumber\\
            x &= \text{displacement} \nonumber\\
            P &= \text{power} \nonumber\\
            \phi &= \text{smallest angle between two vectors} \nonumber
        \end{align}

        \paragraph{Constants}
        \begin{align}
            g &= \SI[per-mode=symbol]{9.8}{\meter\per\square\second} \nonumber
        \end{align}

    \section{Kinetic Energy}

        Kinetic energy is energy associated with the state of motion of an object. 
        \begin{equation}
            K = \frac{1}{2} m v^2
        \end{equation}

    \section{Work and Kinetic Energy}

        \paragraph{Work}
        Work is energy transferred to or from an object by means of a force acting on the object. Energy transferred to the object is positive work, and energy transferred from the object is negative work.

        \paragraph{Work Done by a Constant Force}
        \begin{equation}
            W = \vec{F} \cdot \vec{d} = F d \cos{\phi}
        \end{equation}

    \section{Work Done By the Gravitational Force}

        \begin{equation}
            W_g = m g d \cos{\phi}
        \end{equation}

    \section{Work Done by a Spring Force}

        \paragraph{Hooke's Law}
        Hooke's Law can be used as a general rule to approximate the force using a force constant $k$.
        \begin{equation}
            F_x = -k x
        \end{equation}

        \paragraph{Work Done by a Spring Force}
        \begin{equation}
            W_s = \frac{1}{2} k x_i^2 - \frac{1}{2} k x_f^2
        \end{equation}

    \section{Work Done by a General Variable Force}

        \begin{equation}
            W = \int_{x_i}^{x_f} F(x) \, dx
        \end{equation}

    \section{Power}

        \begin{equation}
            P = \vec{F} \cdot \vec{v}
        \end{equation}

\end{document}
