\documentclass{article}

\title{Chapter 27: Circuits}
\author{Rylan Polster}

\usepackage[parfill]{parskip}
\usepackage{multirow}
\usepackage{amsmath,amssymb,amsthm}
\usepackage{bm}
\usepackage{textcomp,gensymb}
\usepackage{siunitx}
\usepackage[margin=1.0in]{geometry}

\begin{document}
    \maketitle
    
    \section*{General}

        \paragraph{Quantities}
        \begin{align}
            R &= \text{resistance} \nonumber\\
            i &= \text{current} \nonumber\\
            \mathcal{E} &= \text{electromotive force (EMF)} \nonumber\\
            V_\text{AB} &= \text{terminal voltage} \nonumber
        \end{align}

    \section{Single-Loop Circuits}

        \paragraph{Loop Rule}
        The algebraic sum of the changes in potential encountered in a complete traversal of any loop of a circuit must be zero.

        \paragraph{Resistance Rule}
        For a move through a resistance in the direction of the current, the change in potential is $-iR$; in the opposite direction it is $+iR$.

        \paragraph{EMF Rule}
        For a move through an ideal emf device in the direction of the emf arrow, the change in potential is $+\mathcal{E}$; in the opposite direction it is $-\mathcal{E}$.

        \paragraph{Terminal Voltage and Internal Resistance}
        \begin{equation}
            V_\text{AB} = \mathcal{E} - i R
        \end{equation}

    \section{Resistors in Parallel and in Series}
        
        \paragraph{Resistance In Parallel}
        \begin{equation}
            R_\text{eq} = \sum_{j=1}^n \frac{1}{R_j}
        \end{equation}

        \paragraph{Resistance In Series}
        \begin{equation}
            R_\text{eq} = \sum_{j=1}^n R_j
        \end{equation}

    \section{The Ammeter and the Voltmeter}

        \paragraph{Ammeters}
        Ammeters must be placed in series with the component whose current is being measured. The ammeters should have a resistance that is as low as possible ($R_\text{ammeter} \approx \SI{0}{\ohm}$).

        \paragraph{Voltmeters}
        Voltmeters must be placed in parallel with the component whose voltage is being measured. The voltmeter should have a resistance that is as high as possible ($R_\text{voltmeter} \approx \infty$).

\end{document}
