\documentclass{article}

\title{Chapter 9: Center of Mass and Linear Momentum}
\author{Rylan Polster}

\usepackage[parfill]{parskip}
\usepackage{multirow}
\usepackage{amsmath,amssymb,amsthm}
\usepackage{bm}
\usepackage{textcomp,gensymb}
\usepackage{siunitx}
\usepackage[margin=1.0in]{geometry}

\begin{document}
    \maketitle

    \section*{General}

        \paragraph{Quantities}
        \begin{align}
            x_{\text{com}} &= \text{center of mass} \nonumber\\
            m &= \text{mass} \nonumber\\
            M &= \text{total mass of a system} \nonumber\\
            \rho &= \text{density} \nonumber\\
            V &= \text{volume} \nonumber\\
            \vec{v} &= \text{velocity} \nonumber\\
            \vec{v}_{\text{com}} &= \text{velocity of the center of mass} \nonumber\\
            \vec{p} &= \text{linear momentum} \nonumber\\
            \vec{P} &= \text{linear momentum of a system} \nonumber\\
            \vec{F}_{\text{net}} &= \text{net force} \nonumber
        \end{align}

    \section{Center of Mass}

        \paragraph{Systems of Particles}
        $y_{\text{com}}$ and $z_{\text{com}}$ can be calculated by replacing $x_i$ with $y_i$ and $z_i$ respectively.
        \begin{equation}
            x_{\text{com}} = \frac{1}{M} \sum_{i=1}^n m_i x_i
        \end{equation}

        \paragraph{Solid Bodies}
        $y_{\text{com}}$ and $z_{\text{com}}$ can be calculated by replacing $x$ with $y$ and $z$ respectively.
        \begin{equation}
            x_{\text{com}} = \frac{1}{M} \int x \, dm \label{eq:com-integral}
        \end{equation}
        To solve this, use the following relationship:
        \begin{equation}
            \rho = \frac{dm}{dV} = \frac{M}{V} \nonumber
        \end{equation}
        \begin{equation}
            dm = \frac{M}{V} dV \nonumber
        \end{equation}
        Therefore, by substituting the above into \eqref{eq:com-integral} for $dm$:
        \begin{equation}
            x_{\text{com}} = \frac{1}{V} \int x \, dV
        \end{equation}

    \section{Linear Momentum}

        \paragraph{Single Particle}
        \begin{equation}
            \vec{p} = m \vec{v}
        \end{equation}
        \begin{equation}
            \vec{F}_{\text{net}} = \frac{d\vec{p}}{dt}
        \end{equation}
        \paragraph{System of Particles}
        \begin{equation}
            \vec{P} = M \vec{v}_{\text{com}}
        \end{equation}
        \begin{equation}
            \vec{F}_{\text{net}} = \frac{d\vec{P}}{dt}
        \end{equation}

    \section{Collision and Impulse}

        \paragraph{Impulse}
        \begin{equation}
            \vec{J} = \Delta \vec{p} = \int_{t_i}^{t_f} \vec{F}(t) \, dt
        \end{equation}

    \section{Conservation of Linear Momentum}

        If no net external force acts on a system of particles, the total linear momentum $\vec{P}$ of the system cannot change.

    \section{Momentum and Kinetic Energy In Collisions}

        \paragraph{Elastic Collision}
        The total kinetic energy of the system is \textit{conserved} (it is the same before and after the collision).

        \paragraph{Inelastic Collision}
        Some energy is transferred from kinetic energy to other forms of energy, such as thermal energy.

        \paragraph{Completely Inelastic Collision}
        The bodies stick together resulting in the greatest amount of kinetic energy being lost.

\end{document}
