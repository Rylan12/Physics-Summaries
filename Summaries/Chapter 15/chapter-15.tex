\documentclass{article}

\title{Chapter 15: Oscillations}
\author{Rylan Polster}

\usepackage[parfill]{parskip}
\usepackage{multirow}
\usepackage{amsmath,amssymb,amsthm}
\usepackage{bm}
\usepackage{textcomp,gensymb}
\usepackage{siunitx}
\usepackage[margin=1.0in]{geometry}

\begin{document}
    \maketitle

    \section*{General}

        \paragraph{Quantities}
        \begin{align}
            T &= \text{period of oscillation} \nonumber\\
            f &= \text{frequency of oscillation} \nonumber\\
            x &= \text{displacement} \nonumber\\
            x_m &= \text{maximum displacement} \nonumber\\
            \omega &= \text{angular frequency} \nonumber\\
            \phi &= \text{phase angle} \nonumber\\
            v &= \text{velocity} \nonumber\\
            a &= \text{acceleration} \nonumber\\
            F &= \text{force} \nonumber\\
            k &= \text{spring constant} \nonumber\\
            U &= \text{potential energy} \nonumber\\
            K &= \text{kinetic energy} \nonumber\\
            E &= \text{total mechanical energy} \nonumber
        \end{align}

    \section{Simple Harmonic Motion}

        \paragraph{Period of Oscillation}
        \begin{equation}
            T = \frac{1}{f}
        \end{equation}

        \paragraph{Angular Frequency}
        \begin{equation}
            \omega = \frac{2 \pi}{T} = 2 \pi f
        \end{equation}

        \paragraph{Displacement}
        \begin{equation}
            x(t) = x_m \cos{\left( \omega t + \phi \right)}
        \end{equation}

        \paragraph{Velocity}
        \begin{equation}
            v(t) = - \omega x_m \sin{\left( \omega t + \phi \right)}
        \end{equation}

        \paragraph{Acceleration}
        \begin{equation}
            a(t) = - \omega^2 x_m \cos{\left( \omega t + \phi \right)}
        \end{equation}
        \begin{equation}
            a(t) = - \omega^2 x(t)
        \end{equation}
        In SHM, the acceleration $a$ is proportional to the displacement $x$ but opposite in sign, and the two quantities are related by the square of the angular frequency $\omega$.

        \paragraph{Hooke's Law}
        \begin{equation}
            F = - k x
        \end{equation}

    \section{Energy In Simple Harmonic Motion}

        \paragraph{Potential Energy}
        \begin{equation}
            U(t) = \frac{1}{2} k x^2 = \frac{1}{2} k x_m^2 \cos^2 \left( \omega t + \phi \right)
        \end{equation}

        \paragraph{Kinetic Energy}
        \begin{equation}
            K(t) = \frac{1}{2} m v^2 = \frac{1}{2} m \omega^2 x_m^2 \sin^2 \left( \omega t + \phi \right)
        \end{equation}

        \paragraph{Total Mechanical Energy}
        \begin{equation}
            E = U + K = \frac{1}{2} k x_m^2
        \end{equation}

\end{document}
