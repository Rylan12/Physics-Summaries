\documentclass{article}

\title{Chapter 22: Electric Fields}
\author{Rylan Polster}

\usepackage[parfill]{parskip}
\usepackage{multirow}
\usepackage{amsmath,amssymb,amsthm}
\usepackage{siunitx}
\usepackage[margin=1.0in]{geometry}

\begin{document}
    \maketitle
    
    \section*{General}

        \paragraph{Quantities}
        \begin{align}
            \vec{E} &= \text{electric field} \nonumber\\
            \vec{F} &= \text{electrostatic Force} \nonumber\\
            q &= \text{point charge} \nonumber\\
            r &= \text{distance} \nonumber\\
            \epsilon_0 &= \text{vacuum permittivity constant} \nonumber\\
            k &= \text{Coulomb's law constant} \nonumber
        \end{align}

        \paragraph{Constants}
        \begin{align}
            \epsilon_0 &= \SI[per-mode=fraction]{8.85e-12}{\coulomb\squared\per\newton\per\meter\squared} \nonumber\\
            k &= \frac{1}{4 \pi \epsilon_0} = \SI[per-mode=fraction]{9.0e9}{\newton\meter\squared\per\square\coulomb} \nonumber
        \end{align}

    \section{The Electric Field}

        \paragraph{Electric Field}
        The electric field is a vector field with measures of the electrostatic force at a given point in space. The values are calculated using a test charge that is always positive. The units are $\si[per-mode=fraction]{\newton\per\coulomb}$ or $\si[per-mode=fraction]{\volt\per\meter}$.
        \begin{equation}
            \vec{E} = \frac{\vec{F}}{q_0}\\
        \end{equation}
        \begin{equation}
            \vec{F} = \vec{E} \cdot q_0
        \end{equation}

        \paragraph{Electric Field Lines}
        Electric field lines are lines that represent the electric field vector at a given point. They always point toward a more negative charge (or away from a more positive charge).

        At a given point along the electric field line, the electric field vector is tangent to the line.

        The magnitude of the electric field vector is bigger when the electric field lines are farther apart.
    
    \section{The Electric Field Due To A Charged Particle}

        \begin{equation}
            \vec{E} = \frac{\vec{F}}{q_0} = \frac{1}{4\pi\epsilon_0} \frac{\left|q\right|}{r^2} = k \frac{\left|q\right|}{r^2}
        \end{equation}
        \begin{equation}
            \vec{E} = \vec{E_1} + \vec{E_2} + \ldots + \vec{E_n}
        \end{equation}

    \section*{Electrostatic Equation Grid}

        \begin{center}
        \begin{tabular}{|c|c|c|}
            \hline                                      & \textbf{Pair of Charges}                  & \textbf{Point in Space} \\\hline
            \multirow{4}{*}{\textbf{Vector Quantities}} &                                           & \\
                                                        & \(\displaystyle\vec{F} = k \frac{\left|q_1\right|\left|q_2\right|}{r^2}\) & \(\displaystyle\vec{E} = k \frac{\left|q\right|}{r^2}\) \\
                                                        &                                           & \\
                                                        & Electric Force                            & Electric Field \\\hline
            \multirow{4}{*}{\textbf{Scalar Quantities}} &                                           & \\
                                                        & \(\displaystyle U = k \frac{q_1 q_2}{r}\) & \(\displaystyle V = k \frac{q}{r}\) \\
                                                        &                                           & \\
                                                        & Electric Potential Energy                 & Electric Potential \\\hline
        \end{tabular}
        \end{center}

\end{document}
