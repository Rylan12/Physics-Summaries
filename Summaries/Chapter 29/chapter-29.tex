\documentclass{article}

\title{Chapter 29: Magnetic Fields Due to Currents}
\author{Rylan Polster}

\usepackage[parfill]{parskip}
\usepackage{multirow}
\usepackage{amsmath,amssymb,amsthm}
\usepackage{bm}
\usepackage{textcomp,gensymb}
\usepackage{siunitx}
\usepackage[margin=1.0in]{geometry}

\begin{document}
    \maketitle
    
    \section*{General}

        \paragraph{Quantities}
        \begin{align}
            B &= \text{magnetic field} \nonumber\\
            i &= \text{current} \nonumber\\
            \mu_0 &= \text{permeability constant} \nonumber\\
            R &= \text{radius} \nonumber\\
            \phi &= \text{angle of arc} \nonumber\\
            n &= \text{number of turns per unit length of a solenoid} \nonumber
        \end{align}

        \paragraph{Constants}
        \begin{align}
            \mu_0 = \SI[per-mode=fraction]{4 \pi e-7}{\tesla\meter\per\ampere} \nonumber
        \end{align}

    \section{Magnetic Field Due to a Current}

        \paragraph{Magnetic Field Due to a Current in a Long Straight Wire}
        \begin{equation}
            B = \frac{\mu_0 i}{2 \pi R}
        \end{equation}

        \paragraph{Magnetic Field Due to a Current in a Loop of Wire}
        \begin{equation}
            B = \frac{\mu_0 i}{2 R}
        \end{equation}

        \paragraph{Magnetic Field Due to a Current in a Circular Arc of Wire}
        \begin{equation}
            B = \frac{\mu_0 i \phi}{4 \pi R}
        \end{equation}

    \section{Solenoids}

        \paragraph{Magnetic Field Inside a Solenoid}
        \begin{equation}
            B = \mu_0 i n
        \end{equation}

\end{document}
