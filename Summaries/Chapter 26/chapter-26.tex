\documentclass{article}

\title{Chapter 26: Current and Resistance}
\author{Rylan Polster}

\usepackage[parfill]{parskip}
\usepackage{multirow}
\usepackage{amsmath,amssymb,amsthm}
\usepackage{bm}
\usepackage{textcomp,gensymb}
\usepackage{siunitx}
\usepackage[margin=1.0in]{geometry}

\begin{document}
    \maketitle
    
    \section*{General}

        \paragraph{Quantities}
        \begin{align}
            V &= \text{voltage} \nonumber\\
            i &= \text{current} \nonumber\\
            R &= \text{resistance} \nonumber\\
            q &= \text{charge} \nonumber\\
            t &= \text{time} \nonumber\\
            J &= \text{current density} \nonumber\\
            A &= \text{cross-sectional area} \nonumber\\
            L &= \text{length} \nonumber\\
            n &= \text{charge carriers per unit volume} \nonumber\\
            v_d &= \text{drift speed} \nonumber\\
            e &= \text{electron charge magnitude} \nonumber\\
            \rho &= \text{resistivity} \nonumber\\
            \sigma &= \text{conductivity} \nonumber\\
            E &= \text{electric field} \nonumber\\
            P &= \text{power} \nonumber
        \end{align}

        \paragraph{Constants}
        \begin{align}
            e &= \SI{1.60e-19}{\coulomb} \nonumber\\
        \end{align}

    \section{Electric Current}

        \begin{equation}
            i = \frac{dq}{dt}
        \end{equation}

        \paragraph{Units}
        \begin{equation}
            \SI{1}{\text{ampere}} = \SI{1}{\ampere} = \SI{1}{\text{coulomb per second}} = \SI[per-mode=symbol]{1}{\coulomb\per\second} \nonumber
        \end{equation}

        \paragraph{Junctions}
        \begin{equation}
            i_\text{in} = i_\text{out}
        \end{equation}

    \section{Current Density}

        The current density tells how many charges are flowing within a given area of a conductor.

        Units: $\si[per-mode=symbol]{\ampere\per\square\meter}$
        \begin{equation}
            i = \int \vec{J} \cdot d\vec{A}
        \end{equation}

        \paragraph{Uniform Current}
        \begin{equation}
            J = \frac{i}{A}
        \end{equation}

        \paragraph{Total Charge Within Length $L$}
        \begin{equation}
            q = \left( n A L \right) e
        \end{equation}

        \paragraph{Time For A Charge To Move Through The Conductor}
        \begin{equation}
            t = \frac{L}{v_d}
        \end{equation}

        \paragraph{Current} Mnemonic: ``nevada"
        \begin{equation}
            i = \frac{q}{t} = \frac{\left( n A L \right) e}{L/v_d} = n A e v_d
        \end{equation}

        \paragraph{Current Density}
        \begin{equation}
            \vec{J} = \left( n e \right) \vec{v}_d
        \end{equation}

    \section{Resistance and Resistivity}

        \paragraph{Resistance}
        Units: $\si{\ohm}$
        \begin{equation}
            R = \frac{V}{i}
        \end{equation}

        \paragraph{Resistivity}
        Units: $\si{\ohm\meter}$
        \begin{equation}
            \rho = \frac{E}{J}
        \end{equation}
        \begin{equation}
            \vec{E} = \rho \vec{J}
        \end{equation}
        
        \paragraph{Conductivity}
        \begin{equation}
            \sigma = \frac{1}{\rho}
        \end{equation}

        \paragraph{Resistance and Resistivity}
        Resistance is a property of an object. Resistivity is a property of a material.
        \begin{equation}
            R = \rho \frac{L}{A}
        \end{equation}

    \section{Ohm's Law}

        \begin{equation}
            V = i R
        \end{equation}

    \section{Power}

        \begin{equation}
            P = i V
        \end{equation}
        \begin{equation}
            P = i^2 R
        \end{equation}
        \begin{equation}
            P = \frac{V^2}{R}
        \end{equation}

\end{document}
