\documentclass{article}

\title{Chapter 6: Force and Motion — 2}
\author{Rylan Polster}

\usepackage[parfill]{parskip}
\usepackage{multirow}
\usepackage{amsmath,amssymb,amsthm}
\usepackage{bm}
\usepackage{textcomp,gensymb}
\usepackage{siunitx}
\usepackage[margin=1.0in]{geometry}

\begin{document}
    \maketitle
    
    \section*{General}

        \paragraph{Quantities}
        \begin{align}
            F &= \text{force} \nonumber\\
            m &= \text{mass} \nonumber\\
            v &= \text{velocity} \nonumber\\
            a &= \text{acceleration} \nonumber\\
            g &= \text{magnitude of free-fall acceleration} \nonumber\\
            f &= \text{friction force} \nonumber\\
            W &= \text{weight} \nonumber\\
            N &= \text{normal force} \nonumber\\
            D &= \text{drag force} \nonumber\\
            R &= \text{radius} \nonumber\\
            \mu_s &= \text{coefficient of static friction} \nonumber\\
            \mu_k &= \text{coefficient of kinetic friction} \nonumber\\
            C &= \text{drag coefficient} \nonumber\\
            \rho &= \text{air density} \nonumber\\
            A &= \text{cross-sectional area} \nonumber\\
            v_t &= \text{terminal speed} \nonumber
        \end{align}

        \paragraph{Constants}
        \begin{align}
            g &= \SI[per-mode=symbol]{9.8}{\meter\per\square\second} \nonumber
        \end{align}

    \section{Friction}

        \paragraph{Static Friction}
        \begin{equation}
            f_{s\text{,max}} = \mu_s N
        \end{equation}

        \paragraph{Kinetic Friction}
        \begin{equation}
            f_k = \mu_k N
        \end{equation}

    \section{The Drag Force and Terminal Speed}

        \paragraph{Drag}
        \begin{equation}
            D = \frac{1}{2} C \rho A v^2
        \end{equation}
        
        \paragraph{Terminal Speed}
        \begin{equation}
            v_t = \sqrt{\frac{2 W}{C \rho A}}
        \end{equation}

    \section{Uniform Circular Motion}

        \paragraph{Centripetal Acceleration}
        \begin{equation}
            a = \frac{v^2}{R}
        \end{equation}

        \paragraph{Centripetal Force}
        A centripetal force accelerates a body by changing the direction of the body's velocity without changing the body's speed.
        \begin{equation}
            F = m \frac{v^2}{R}
        \end{equation}

\end{document}
